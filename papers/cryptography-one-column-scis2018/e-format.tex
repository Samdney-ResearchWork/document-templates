% SCIS 2018 Manuscript Submission Guidelines(LaTex)
\documentclass[a4paper]{article}
\usepackage{scis2018e}
% If LaTeX 2.09 is used, the above two lines should be deleted
% while the % of the following line should be removed.
%\documentstyle[scis2018e]{article}

\begin{document}

\title{
  How to Format and Submit Your Manuscript for SCIS2018
}

\author{
  SCIS2018 Secretariat\thanks{
    SCIS2018 Secretariat,
    1753, Shimonumabe Nakahara-ku
    Kawasaki, Kanagawa 211-8666, Japan. %
    isec-scis2018@mail.ieice.org}
  \and
  Kaitaro Genko\samethanks{1}
}

\abstract{ % Abstract
  In this guide, we explain the method of preparing and submitting the manuscript for SCIS2018 (online abstract and full paper).
  Text style short online abstracts are scheduled to be published through website and USB memory, and a collection of full papers is scheduled to be distributed at the symposium as ``abstracts of SCIS2018".
  Authors are required to prepare both ``Online abstract within 1,300 characters (in English) or within 500 characters (in Japanese)",
and ``Full paper within eight pages".
The online abstract is registered by text input, while the full paper is submitted by uploading an electronic file. {\bf It is not necessary to submit the hard copy.}
  The registered ``Online abstract" and the submitted ``Full paper" are collected for the Web publishing and the USB-memory abstracts, respectively.
  In order to avoid troubles, please prepare and submit your manuscript according to this guide. }
\keywords{ % keywords
  SCIS 2018, manuscript style, file format. }

\maketitle

\section{Online Abstract}

The online abstract of your manuscript will be published on Web and also included in the USB memories that will be distributed at the symposium.
Please input the text of the online abstract via the online submission system.
It must be {\bf within 1,300 characters (in English)} or {\bf within 500 characters (in Japanese)}.

 After participation application on Web, you can access ``my page". After registering for new presentation, you can input the online abstract under
 ``Manuscript (online abstract)"

\section{Full Paper (USB-memory abstracts)}

USB-memory abstracts will be distributed in SCIS 2018. Please prepare your full paper {\bf not exceeding 8 pages}.

Since the submitted manuscript will be published as it is, please prepare the manuscript according to the following.
Note that this document itself follows the recommended style for the full paper abstracts.

\subsection{Limitations of Document Size and Page Count}

The paper size is set to be $A4$. The manuscript (main text for USB-memory abstracts) must not be more than eight pages.

Please include the title, authors, affiliation, address, abstract, keywords, main text, diagrams, references, and appendices, etc. in the eight pages.

\subsection{Layout of Manuscript}

Please arrange the margins as follows: top margin $17~ mm$, bottom margin $20~ mm$, and the right and left margins $18~ mm$, respectively. Except these margins,
please put the context in an area of height $260~ mm$ and width $174~ mm$.

Moreover, please put the label of SCIS of height $16~ mm$ and width $57~ mm$ in the upper right corner of the first page. For the content and the style,
please follow the label of SCIS in this document. It is not necessary to put the label after the first page. Paper number is also unnecessary.

Please {\bf put page numbers in} the paper.

\subsection{Composition of the 1st Page}

The necessary items and languages for the first page of the text are summarized in Table 1.

\begin{small}
\begin{table}[htbp]
  \begin{center}
    %\leavevmode
    \caption{Necessary items and languages on the 1st page}




      \begin{tabular}{l|c|c} \hline
      Paper Language   & Japanese & English \\
      \hline
      Title            & J and E  &   E     \\
      Authors          & J and E  &   E     \\
      Affiliation      & J and E  &   E     \\
      Address          & J and E  &   E     \\
      Abstract         & J or E   &   E     \\
      Keywords         & J or E   &   E     \\
      \hline
    \end{tabular}
    \label{tab:const}
  \end{center}
      J: Japanese; E: English\\
      J and E: Japanese and English is required \\
      J or E: either Japanese or English\\
      E: only English is required

\end{table}
\end{small}


The ``Abstract" in the full paper need not be the same as the online abstract (but they can be the same).

\subsection{Style File for \LaTeX}

For those who use \LaTeX\ to prepare the manuscript,
the \LaTeX\ style file and the template file were prepared. Please acquire them from \\ \mbox{http://www.iwsec.org/scis/2018/call.html}.\\
There are both English version and Japanese version.

%---------------------------------------------
\section{About Submitting the Manuscript}

\subsection{Presentation Application}

For the presentation application,
please register via the web page, under ``Registration" in \\ \mbox{http://www.iwsec.org/scis/2018/registration.html}.\\
Then proceed to the registration of new presentation after login into ``my page".

\subsection{Manuscript Registration and Submission}

Please submit the online abstract and the full paper online. {\bf It is not necessary to submit the hard copy.}

Please login to ``my page", in order to register and submit the manuscript through the submission page.

Deadline for submission is 10:00, December 18, 2017 (JST). 
After the manuscript is submitted, it can be updated for any number of times until the deadline.
The online abstracts and the USB-memory abstracts are based on the last version of any information submitted.
Early submission is recommended because congestion in the server is expected in the last minutes of the deadline.
Please be informed that if authors fail to submit enough information necessary for organizing the symposium program
by deadline, then the submission may be excluded from the program.


\subsection{File Format}

The online abstract must be {\bf in text form} and the full paper must be in {\bf PDF form}. Moreover, the file size of the full paper must not exceed $1.5MB$ for the convenience
of making the USB memory. If your file is more than $1.5MB$, it may not be put in the USB memory. Please take extra care if the file contains many photographs.

Please also note the following section ``About Fonts That Can Be Used".

\subsection{About Fonts That Can Be Used}

The PDF file of the submitted manuscripts are collected to the USB-memory abstracts ``as it is". Therefore, in order for it to be displayed correctly in different platforms,
the following guideline is recommended.

\begin{description}
 \item[Latin font]
It is safer to embed all the fonts used.
 \item[Japanese font] It is also recommended to embed the Japanese font in the PDF file as long as the file size is not over the limit.
Otherwise, please confirm that the file can be correctly displayed and printed with various sofware when the font is not embedded.
If you use the Windows environment, please check that it also works with other environments such as Mac and UNIX, etc.
\end{description}

Please refer to the manual of your PDF conversion software about font embedding.

\section{Inquiries}

\begin{tabular}{l}
%[
SCIS2018 Secretariat \\
{\small Email: isec-scis2018@mail.ieice.org} \\
\end{tabular}

%\begin{thebibliography}{9}
%\bibitem{a}
%Author name, `` Title of paper, " Journal name, pages, etc.
%\bibitem{b}
%Author name, `` Title of book, " Press name, pages, publishing year etc.
%\end{thebibliography}

\end{document}
% end of file
