% Template for Electronic Journal of Graph Theory and Applications
% This file must be in the same folder as EJGTAart.sty and EJGTAhead.jpg

\documentclass[12pt]{elsarticle}
\usepackage[left=1in,right=1in, top=1.2in,bottom=1.2in]{geometry}
\usepackage{EJGTAart}
\usepackage{times}
\usepackage{amssymb,amsthm,latexsym,amsmath,epsfig,pgf}

% Set the volume if you know.
\volume{{\bf x} (x)}

% Set the starting page
\firstpage{1}

% Title (or short title) and author name for the header
\runauth{
Title of the manuscript
\hspace{2ex} $\arrowvert$\hspace{2ex}
Edy Tri Baskoro et al.}


% Put new theorem or any other settings for your document here
\newtheorem{theorem}{Theorem}[section]
\newtheorem*{theorem A}{Theorem A}
\newtheorem*{theorem B}{N\"olker's Theorem}
\newtheorem{lemma}{Lemma}[section]
\newtheorem{proposition}{Proposition}[section]
\newtheorem{corollary}{Corollary}[section]
\newtheorem{problem}{Problem}
\newtheorem*{question}{Question}
\newtheorem {conjecture}{Conjecture}
\theoremstyle{remark}
\newtheorem{remark}{Remark}[section]
\theoremstyle{remark}
\newtheorem{remarks}{Remarks}
\begin{document}

\begin{frontmatter}
% Write your paper title here
\papertitle{Title of the Manuscript}

%% use optional labels to link authors explicitly to addresses. The label may be more than one, use comma to separate

%\author[label1,label]{Edy Tri Baskoro${}^1$}

\author[label1]{Edy Tri Baskoro}
\author[label2]{Joe Ryan}
\author[label3]{Kiki A. Sugeng}

\address[label1]{\small Combinatorial Mathematics Research Group,\\
Faculty of Mathematics and Natural Sciences, Institut Teknologi Bandung,\\
Jalan Ganesa 10 Bandung, Indonesia}
\address[label2]{School of Electrical Engineering and Computer Science,
The University of Newcatle, Australia}
\address[label3]{Departments of Mathematics,
University of Indonesia,
Depok - Indonesia

\vspace*{2.5ex} 
 {\normalfont ebaskoro@math.itb.ac.id, joe.ryan@newcastle.edu.au, kiki@sci.ui.ac.id}
 }

\begin{abstract}
Electronic Journal of Graph Theory and Apllications ({\bf EJGTA}) is a
fully-refereed electronic journal. ({\bf EJGTA}) is devoted to the
high quality publication of current research developments in the
fields of graph theory, and in all interdisciplinary areas in
mathematics which use graph methods. 

\let\thefootnote\relax\footnotetext{Received: xx xxxxx 20xx,\quad
  Accepted: xx xxxxx 20xx.\\[3ex]
  }
  
\end{abstract}

\begin{keyword}
% Separate keyword by \sep
separate \sep keyword \sep by this \sep command

% Write the classification number
Mathematics Subject Classification : xxxxx

\end{keyword}

\end{frontmatter}

%% Main text
\section{This is a numbered first-level section head}
This is an example of a numbered first-level heading.
\subsection{This is a numbered second-level section head}
This is an example of a numbered second-level heading.

\subsection*{This is an unnumbered second-level section head}
This is an example of an unnumbered second-level heading.

\subsubsection{This is a numbered third-level section head}
This is an example of a numbered third-level heading.
\begin{lemma}
Example of the lemma
\end{lemma}

\begin{remark}
This is an example of a remark element.
\end{remark}

\begin{theorem}
This is an example of a theorem.
\end{theorem}

\begin{theorem}[XXX Theorem]
This is an example of a theorem with a parenthetical note in the
heading.
\end{theorem}


\section*{Acknowledgement} 
Write support, acknowledgment, dedicatory, and grants here.

\section*{References} 
\begin{thebibliography}{99}
\bibitem{Anholcer-309-09} M. Anholcer, M. Kalkowski and J. Przybylo, A new upper bound for the total vertex irregularity strength of graphs, \textit{Discrete Math.} \textbf{309} (2009), 6316--6317.
\bibitem{Brandt-57-08} S. Brandt, J. Mi\v{s}kuf and D. Rautenbach, On a~conjecture about edge irregular total labellings, {\it J. Graph Theory}, \textbf{57} (2008), 333--343.
\bibitem{Chartrand-64-88} G. Chartrand, M.S. Jacobson, J. Lehel, O.R. Oellermann, S. Ruiz and F. Saba, Irregular networks, \textit{Congr. Numer.} \textbf{64} (1988), 187--192.
\bibitem{Ivanco-26-06} J. Ivan\v{c}o and S. Jendro\v{l}, Total edge irregularity strength of trees, {\it Discussiones Math. Graph Theory}  {\bf 26} (2006), 449--456.
\bibitem{Jendrol-28-07} S. Jendro\v{l}, J.Mi\v{s}kuf and R. Sot\'{a}k, Total edge irregularity strength of complete and complete bipartite graphs, {\it Electron. Notes Discrete Math.}  {\bf 28} (2007), 281--285.
\bibitem{Jendrol-310-10} S. Jendro\v{l}, J.Mi\v{s}kuf, and R. Sot\'{a}k, Total edge irregularity strength of complete graphs and complete bipartite graphs, {\it Discrete Math.}, {\bf 310} (2010), 400--407.
\end{thebibliography}

\end{document}

%%
%% End of file `ecrc-template.tex'. 