%%%%%%%%%%%%%%%%%%%%%%%%%%%%%%%%%%%%%%%%%%%%%%%%%%%%%%%%%%
% preamble.tex
% template-preamble
%
% Author: Carolin Zöbelein
% Email: contact@carolin-zoebelein.de
% PGP: D4A7 35E8 D47F 801F 2CF6 2BA7 927A FD3C DE47 E13B
%%%%%%%%%%%%%%%%%%%%%%%%%%%%%%%%%%%%%%%%%%%%%%%%%%%%%%%%%%
\documentclass{scrartcl}

\usepackage[utf8x]{inputenc}
\usepackage[english]{babel}

\usepackage{footnote}
\usepackage{amssymb}
\usepackage{url}
\usepackage{graphicx}
\usepackage{amsmath}
\usepackage{hyperref}
%\usepackage{minitoc}
\usepackage{float}
\usepackage{longtable}
\usepackage{enumitem}

\usepackage{verbatim}	% For block comments

\usepackage{listings}
\lstset{
language=Python,
basicstyle=\small\sffamily,
%basicstyle=\tiny\sffamily,
numbers=left,
numberstyle=\tiny,
frame=tb,
%frame=single,
columns=fullflexible,
showstringspaces=false
}

\newtheorem{theorem}{Theorem}[section]
\newtheorem{lemma}[theorem]{Lemma}
\newtheorem{definition}[theorem]{Definition}
\newtheorem{example}[theorem]{Example}
\newtheorem{xca}[theorem]{Exercise}
\newtheorem{remark}[theorem]{Remark}

% Example: \authoremail{example@example.com, AAAA BBBB CCCC DDDD EEEE FFFF GGGG HHHH IIII JJJJ}
\newcommand{\authoremail}[2]{\textit{E-mail address:} \texttt{#1}, \textit{PGP Fingerprint}: \texttt{#2}}

% Example: \authorurl{http://www.example.com}
\newcommand{\authorurl}[1]{\textit{URL:} \url{#1}}

% Example: \keywords{keyword1, keyword2, keyword3}
\newcommand{\keywords}[1]{\textbf{Keywords:} #1}

% Example: \license{example license}
\newcommand{\license}[1]{\textbf{License:} #1}

% Example: \subjclass{2010}{Mathematics Subject Classification}{Primary 11N05}
\newcommand{\subjclass}[3]{\textbf{Subjclass:} #1 \textit{#2}. #3.}
